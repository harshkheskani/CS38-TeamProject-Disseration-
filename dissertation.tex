% This example An LaTeX document showing how to use the l3proj class to
% write your report. Use pdflatex and bibtex to process the file, creating 
% a PDF file as output (there is no need to use dvips when using pdflatex).

% Modified 

\documentclass{l3proj}

\begin{document}

\title{Leidos Roll Vans Project}

\author{Filip Vyrostko \\
        Alen Reji \\
        Ewan Bell \\
        Harsh Kheskani Raisinghani}

\date{March 2022}

\maketitle

\begin{abstract}

In this paper we present a case study of a software development processes of the Roll Vans web application created for Leidos company as part of Team Project 3 coursework.\\\\
We describe the process itself as well as practises used during its development.\\\\
The conclusion describes our most prominent learning achievements. Moreover, we compare existing software engineering practices with those employed by us during the process.

\end{abstract}

%% Comment out this line if you do not wish to give consent for your
%% work to be distributed in electronic format.
\educationalconsent

\newpage

%==============================================================================
\section{Introduction}

Software engineering 

This paper presents a case study of... 


%% Final paragraph.
The rest of the case study is structured as follows.  Section
\ref{sec:background} presents the background of the case study
discussed, describing the customer and project context, aims
objectives and agreements and project state at the time of writing. Sections
\ref{sec:dev_process} discusses specific aspects of the development process employed by our team as well as documents encountered challenges and our resulting improvements. Lastly, section \ref{sec:deployment} describes deployment, the final step of our development life-cycle. 

%==============================================================================
\section{Case Study Background}
\label{sec:background}
    This section provides background for the Leidos (section \ref{sec:client}), its initial product proposal (section \ref{sec:proposal}), our changes to that proposal, as well as our implementation choices (section \ref{sec:imp_and_agr}). Lastly, in section \ref{sec:delivered}, we discuss our delivered product and its intended use by the customer.
    
    %------------------------------------------------------------------------------
    \subsection{The Client}
    \label{sec:client}
        Leidos is small organization looking for a way to bring small, independent businesses into marketing spotlight without the use of the convectional marketing platforms, all without a fee.
        
    %------------------------------------------------------------------------------
    \subsection{Project Proposal}
    \label{sec:proposal}
        The main idea behind this project was to introduce a platform for small, independent businesses, in order for them to create an online presence. Reason for this is that popular marketing platforms such as Google are simply not an option due to the price for the marketing as well as large competition.\\
        \newline
        The initial proposal suggested an Android and/or iOS mobile application, with integrated maps API to overlay locations of businesses on a map. On top of that, it was required for each businesses to be able to create their on profile with appropriate information such as name, address, description, menu and opening times. Moreover, application should support push notifications for subscribed locations, allow notifications snooze for varying time periods, be lightweight in its performance and requirements and use modern UI design.\\
        Lastly, the proposal hinted on possible inclusion of rating/review system as well as integrated social media sharing and UI space dedicated to advertising for revenue purposes.
        %------------------------------------------------------------------------------
    \subsection{Implementation Goals and Agreement}
    \label{sec:imp_and_agr}
        After the first customer meeting, it was agreed that despite the initial proposal of Android/iOS mobile application, our team will develop a web application instead. This decision was taken due to our concerns of no prior knowledge of mobile application development.
        \newline
        Due to change in platforms, our team decided to create new list of requirement (more on the process in section \ref{sec:req_and_feat}). We have decided to drop mobile application specific features such as 'support for push notifications for subscribed locations' or 'allow notifications snooze for varying time periods'. Nonetheless, we were able to retain most of the features from the original proposal. After getting a green light from the customer on the new list of features, we have presented them with framework intended for the development as well as arguments for the choice. We have decided to use \textit{Django} (web application framework) \cite{Django} framework with its primary language being \textit{Python}.\\
        \newline
        Looking at the choice of framework now it shows our inexperience in field of web development, as using \textit{Rest Framework}\cite{rest} for Django with \textit{Angular}\cite{angular} or \textit{React}\cite{react} as front-end frameworks would have been a much better choice.
        
    %------------------------------------------------------------------------------
    \subsection{Delivered Application}
    \label{sec:delivered}
        Our delivered web application will not be deployed publicly. The main purpose was to create a prototype for the customer to use as a stepping stone in further development, as well as to see if this idea had any potential (will be presented to some real users). 






%==============================================================================
\newpage
\section{Development Process}
\label{sec:dev_process}
    In this section we give insight into our development process. Namely, our interpretation and formulation of requirements and features in section \ref{sec:req_and_feat}, our methods of customer communication and structure of customer meetings in section \ref{sec:communication}, as well as our post meeting reflections and practices in section \ref{rec:retrospectives}.\\
    From section \ref{sec:versio_control} to section \ref{sec:tests} we describe our development practices such as Issue Tracking, Mob and Pair Programming, Code Reviews, CI/CD Pipelines, Tests and Coverage.
    
    %------------------------------------------------------------------------------
    \subsection{Forming Requirements and Features}
    \label{sec:req_and_feat}
        As already mentioned in section \ref{sec:imp_and_agr}, we had to create our own list of features and their priorities on top of the list proposed by the customer. Both original and modified requirements lists were very specific. To better outline the functionality, we have decided to use user stories. As per the proposal, two main user figures arose, a customer and a business owner. Each story focus on one of the two figures, describing their needs and wishes. By constructing these stories, we were able to effectively establish required functionalities and reasoning behind them.
        \newline
        \newline
        After forming the list of features, we decided to create a priority list in order to identify what we believed should be focused on first. For this, we have decided to use MoSCoW\cite{moscow} method as prioritization technique. This prioritization was later presented to the customer and after their agreement, we have proceeded to use planing poker to estimate time to develop for each feature (more in section \ref{sec:versio_control})
        \newline
        \newline
        Our conclusion is that our user stories proofed to be very beneficial in defining features from the end-user point of view and solidifying our reasoning behind each feature. Similarly, MoSCoW principle also proofed to be crucial in defining priorities for each feature and together with techniques mentioned later, helped us create effectively create an efficient timetable for our development. Despite this, in the later stage of the project, we had to minimise the scope change in favor of retaining and polishing certain features. As discussed in article by Kelly Hunsenberg \cite{change}, scope change can be beneficial. We can certainly agree, if only for the fact it gave us good lesson on how to better refine our priority management in the future. 
    
    \newpage
    %------------------------------------------------------------------------------
    \subsection{Customer Communication and Meetings}
    \label{sec:communication}
        Early on in our project we have been advised to consider client-team communication a crucial part of our development process. In order to address this part of the process, following the Agile principle of \textit{delivering working
        software frequently}\cite{manifesto}, we split our project into 6 iterations. Each iteration lasting roughly a month, at end which we presented customer with our progress.\\
        Later in the development process, we managed to implement CI/CD pipeline (more in section \ref{sec:cicd}) which further enhanced the already mention Agile principle. Furthermore, it gave us the opportunity to employ another Agile principle, namely the principle of \textit{changing requirements}\cite{manifesto}, as we were able to receive constant feedback from the deployed application, which sometimes changed the requirements for the sprint in progress.\\
        \newline
        Each iteration (sprint) meeting focused on the presentation of the progress since the last meeting and the current state of the project. This was fallowed by a feedback from the customer with a discussion on set of deliverables for the next sprint.
        
        
        \begin{table}[H]
        \begin{tabular}{ll}
        \textbf{Date}               & \textbf{Meeting Purpose}         \\
        October 13th 2021  & Initial Project Meeting \\
        November 10th 2021 & Iteration 1 Summary     \\
        December 1st 2021  & Iteration 2 Summary     \\
        Januray 19th 2022  & Iteration 3 Summary     \\
        February 16th 2022 & Iteration 4 Summary     \\
        March 23rd 2022    & Final Customer Day   
        \end{tabular}
        \quad
        \begin{tabular}{ll}
        \textbf{Time}   & \textbf{Purpose}                                 \\
        5  min  & Introduction to Meeting Content         \\
        10 min & Demonstration of Our Progress           \\
        7  min  & Customer Feedback                       \\
        3  min  & Present Deliverables for Next Iteration \\
        5  min  & Summarize the Meeting                  
        \end{tabular}
        \caption{Left table: Client meetings    Right table: Meeting Timeboxing}
        \end{table}
        
        \newline
        \newline
        More specifically, our first meeting was used to communicate and present customer with our changed proposal for the meeting [\ref{sec:req_and_feat}] and technology and licensing choices. Our second customer meeting was used to communicate our conceptual visualization of the proposal via \textit{Figma} Wireframe. Wireframe was used as to utilize already mentioned concept of \textit{delivering working software frequently}\cite{manifesto}, as it was decided the best way to deliver a mock up of the web application. Wireframes provide quick and relieble way of structuring such mock up and respond well to sudden \textit{requirement changes}\cite{manifesto}\cite{wireframe}.
        \newline
        Our two other communication channels were Microsoft Teams and emails in case direct communication in between iterations was needed.
        
        \newline
        \newline
        Despite good results with client-team communication, we had neglected good communication internally, causing one of the features to be delayed into the next sprint. Section \ref{rec:retrospectives} and section \ref{sec:mob} touch on this subject in more details.
        
        \newline
        \newline
        Overall, these practices greatly helped us to reduce any miscommunication from happening. However, we have to admit  our customers were good in communicating their requirements. Nonetheless, there were few incidents where direct communication with customer was very slow. For example, when deciding on colouring of the page, email was sent to customers asking for their input. The response came fairly late, however we were able to adapt to it fairly quickly. As for meetings themselves, we have implemented strict time-boxing in the later part of the development as a result of one of our retrospectives[\ref{rec:retrospectives}]. This improved the value of meetings for us, as they were more on point, thus reducing miscommunication or discussing topics of a lesser priority.\\
        In retrospect, if we were to make some changes to our communication and planning, we would probably split last two iterations into 4 as suggested by extreme programming methods\cite{ep} and sought to establish more frequent and stable communication loops.
        
        
    %------------------------------------------------------------------------------
    \subsection{Post Meeting Reflection}
    \label{rec:retrospectives}
        Talk about our retrospectives, 5 whys, Start Stop Continue, how they improved our dev. process.
        
    %------------------------------------------------------------------------------
    \subsection{Issue Tracking}
    \label{sec:versio_control}
        Version control. How commit messages were structured. How merge requests were made.
        
    %------------------------------------------------------------------------------
    \subsection{Mob/Pair Programming and Code Reviews}
    \label{sec:mob}
        Talk about how we held weekly programming and code review sessions (after specific sprint). Benefits it gave us.
        
    %------------------------------------------------------------------------------
    \subsection{CI/CD Pipeline}
    \label{sec:cicd}
        Talk about the initial problems of setting CI/CD pipeline, how we later managed to have it working and used it to our benefit.
        
    %------------------------------------------------------------------------------
    \subsection{Tests and Coverage}
    \label{sec:tests}
        Talk about creation of unit tests, about our initial test -> implementation approach that was scrapped. Test coverage using coverage package.




% ==============================================================================
\section{Deployment Process}
\label{sec:deployment}
    Talk about initial Heroku deployed app but then moving to PythonAnywhere due to the technical issues we had (media files). Inability to use CI/CD pipeline for PythonAnywhere
    
    %------------------------------------------------------------------------------
    \subsection{Licensing and Handover Process}

%==============================================================================
\section{Conclusions}

Explain the wider lessons that you learned about software engineering,
based on the specific issues discussed in previous sections.  Reflect
on the extent to which these lessons could be generalised to other
types of software project.  Relate the wider lessons to others
reported in case studies in the software engineering literature.

%==============================================================================
\bibliographystyle{plain}
\bibliography{dissertation}
\end{document}
